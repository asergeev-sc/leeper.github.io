\documentclass[12pt]{article}
\usepackage{hyperref, tabularx, booktabs, mdwlist, verbatim}
\usepackage[top=1in, bottom=1in, left=1in, right=1in]{geometry}
\usepackage[sc,osf]{mathpazo}
\hypersetup{
    bookmarks=false,         % show bookmarks bar?
    pdfstartview={FitH},    % fits the width of the page to the window
    pdftitle={Teaching Portfolio},    % title
    pdfauthor={Thomas J. Leeper},     % author
    pdfsubject={Curriculum Vitae},   % subject of the document
    pdfkeywords={Thomas J. Leeper} {Leeper} {Teaching} {Teaching Portfolio}, % list of keywords
    pdfnewwindow=true,      % links in new window
    pdfborder={0 0 0}
}

\begin{document}
\title{Teaching Portfolio}
\author{Thomas J. Leeper\\
Department of Political Science and Government\\
Aarhus University}

\maketitle

\tableofcontents
\clearpage

\section{Philosophy and Motivation}
Political psychologists study people. We want to know how people think, act, and understand the social and political universe. It is this curiosity about human understanding of politics that fundamentally drives my experience as an educator. Just as great teachers cultivated my interests in politics, I see it as my responsibility to mentor others in developing an understanding for themselves of political questions and the happenings of politics. My research typically explores how people think and feel about politics and often grapples with meta-scientific questions about how researchers might come to know individuals' political cognitions, the causes of those beliefs and attitudes, and the implications thereof for politics more broadly. Teaching is a natural transition from this exploration of human psychology and behavior in which the objective is not for me to transmit information and formulate students' understanding but to guide them as they seek answers to their own political questions.

In teaching political science, it is my objective to leave students not only with a richer understanding of politics but also with conceptual and analytic tools they can apply to critically receive information about the broader social world. I believe it is vital to offer students something more than mere exposure to academic Political Science; what they are taught and what they learn from my courses must be applicable in their future lives regardless of their chosen careers. Toward this end, I see the encouragement of individual choice in topical exploration, the cultivation of useful methodological, analytic, and communication skillsets, and persistent return to everyday social and political examples as vital elements of undergraduate education. %In this way, teaching political science must serve to both train students in the use of a diverse set of tools for understanding the world --- quantitative methods, qualitative data collection, formal theory, normative and textual critique, etc. --- and expose students to the simple facts --- facts clarified by consideration of real political happenings --- that politics is complicated, little there is clear cut, and power operates in every facet of the social world.

Most important of all, however, I see the integration of pedagogy and research as a core means of actively engaging students in the analytic thinking and professional development that are critical to their future success.
%I hope to share my research with future students and involve them in my own discovery of new knowledge, like the faculty that inspired me during my undergraduate and graduate experiences, as well as mentor and guide future students' own explorations into the political and social questions that motivate and excite them. This reciprocal transmission of assistance seems to me an underappreciated opportunity to provide students with opportunities for professional development as well as fully embed advanced students in research activities, professional networking, and technical skill development that will differentiate them in later employment.
Formulating interesting, applied research questions, translating those questions into constructs, operational definitions, and analysis (be it qualitative or quantitative), and communicating findings aurally, visually, and in written form for diverse audiences are precisely the skills that differentiate well-qualified college graduates. To the extent that students additionally obtain nuanced understandings of complex political, social, and psychological phenomena as a result, research experience is doubly beneficial. Politics is, therefore, a means to motivate thinking and a context in which students can develop competence in these skill areas.

I see my role in these aspects of learning as simultaneously a benevolent critic with high expectations and as a resource with expertise on the political and methodological matters that students and I can explore together. %Critical to my teaching style is a persistent availability, both in-person and online, so that students can readily pose and receive answers to questions as well as discuss and elaborate their ideas about course materials.
%In this way, the route I see to great teaching is the same political curiosity that drives my own research matched with both a recognition that many students have little prior knowledge of politics or empirical research.
I have taught students --- through a heavy emphasis on interactive sections, use of office hours, and extensive written feedback --- to never be satisfied with simple answers to simple questions. My goal in previous courses --- as it will be in future courses --- is to push students to generate provocative research topics and then drive them to look beyond ``mono-causal'' answers to those questions. By emphasizing written feedback to written work, I believe students simultaneously learn to wrestle with the substance of complex ideas and the often more difficult task of translating those complexities into concise, comprehensible expressions of ideas.

\clearpage

\section{Teaching Experience}

At Aarhus, I have served as instructor on three seminars, all of which have been designed to advance my general approach to teaching as outlined above. Here, I briefly list all of my relevant teaching experience and, in the next section, I provide additional details on the design of some specific courses.

\begin{itemize*}
\item Instructor, Experimentation and Causal Inference
	\begin{itemize*}
	\item Master seminar, Aarhus, Spring 2014
	\item Responsibilities: Course design, lectures, student advising, and grading
	\item Exam: Home assignment
	\end{itemize*}
\item Co-Instructor, Quantitative Political Analysis
	\begin{itemize*}
	\item Master and PhD seminar, Aarhus, Spring 2014
	\item Responsibilities: Course design, lectures, weekly tutorial sessions, and student advising
	\item Exam: Seven-day home assignment and short assignments
	\end{itemize*}
%\item Instructor, Limited Dependent Variable Regression Models
%	\begin{itemize*}
%	\item PhD short course, Aarhus, Spring 2014 (Planned)
%	\item Responsibilities: Course design and workshop-format teaching sessions
%	\end{itemize*}
\item Instructor, Does Public Opinion Matter?
	\begin{itemize*}
	\item Master seminar, Aarhus, Fall 2013
	\item Responsibilities: Course design, instruction, and grading
	\item Exam: Seven-day home assignment
	\end{itemize*}
\item Teaching Assistant, Statistical Research Methods
	\begin{itemize*}
	\item Bachelor lecture, Northwestern, Spring 2012
	\item Responsibilities: Weekly tutorial sessions, grading, and occasional lecturing
	\item Exam: Written home assignments and two-hour written exam
	\end{itemize*}
\item Teaching Assistant, Methods of Political Inference
	\begin{itemize*}
	\item Bachelor lecture, Northwestern, Spring 2011
	\item Responsibilities: Weekly tutorial sessions and grading
	\item Exam: Home assignments
	\end{itemize*}
\item Teaching Assistant, Political Psychology of Mass Behavior
	\begin{itemize*}
	\item Bachelor lecture, Minnesota, Spring 2007
	\item Responsibilities: Student consulting and grading
	\item Exam: Two-hour written exam
	\end{itemize*}
\end{itemize*}

Aside from my classroom-based experiences, I have informally advised numerous junior doctoral students on coursework, methodological skills, masters theses, and early dissertation development. I have also informally offered guidance on statistical topics and the R statistical programming language through my PhD and postdoc tenures. More formally, I have taught a one-day faculty and PhD workshop on R at Aarhus in Fall 2014. I plan to continue these informal advising roles as well as being to supervisor masters thesis projects in the coming years.

\clearpage
\section{Teaching Preparations}



\subsection{Examples of Course Design and Instruction}

As examples of how my broad teaching philosophy translates into concrete didactic choices, I outline the design of two courses that I have taught (or will be teaching) at Aarhus University during the 2013--2014 academic year. 

\subsubsection{``Does Public Opinion Matter?''}

The purpose of this course is to explore issues related to public opinion --- what opinions are and how they are formed, how opinions shape citizens' political behavior, and how legislatures and other governmental institutions respond (or do not respond) to citizens' preferences. Students will leave the course with a thorough theoretical understanding of political opinions, their origins, and their possible effects through exposure to philosophical perspectives, contemporary case studies, and a broad set of empirical research. The course will challenge assumptions about what democracy is and how it works, but will also provide students with insight into how government --- in legislative, judicial, and bureaucratic capacities --- should work and what role public servants have in influencing and responding to the public's views. The intended learning outcomes for the course are as follows:

\begin{enumerate*}
\item Explain what opinions are, how they are formed, and how they behave.
\item Apply knowledge of opinions and opinion measurement to the evaluation of survey public opinion research.
\item Explain different conceptualizations of political representation and their empirical implications.
\item Apply theories of representation to the evaluation of public processes and institutions.
\item Evaluate arguments about the proper role of public opinion in democracy and government.
\end{enumerate*}

Toward this end, students each week read a number of readings on a specific topic and two students write short essays that are used to catalyze in-class discussions of those readings. Learning activities throughout the course vary, but balance short lectures for the full group, small group discussions, full-group debates in structured and unstructured formats, and peer feedback on short essays in preparation for the exam. As an example, one week students read a short text and then were randomly assigned to write a short paper before class that either defended or challenged the text. Students then met during class in groups to agree upon their strongest and weakest arguments, using other theory and empirical material as evidence. Students then debated the text, allowing them to apply their theoretical knowledge to the specific case, as well as learn how to explain different theoretical perspectives. Students are assessed via a seven-day home assignment, for which these activities have well-prepared them.

\begin{comment}%
\subsubsection{``Experimentation and Causal Inference''}

The purpose of this course is to introduce and elaborate identification-oriented research methods, particularly experimentation, and their use in the social sciences. The focus is on delivering a breadth of substantive research topics and methodological considerations that emerge in experimentation, as well as issues in the analysis and reporting of experimental research, such as matters of validity, mediation and moderation, treatment noncompliance, and the use of covariates. Students will leave the course with a deep and broad understanding of experimental design, along its challenges and opportunities, in line with the following learning outcomes: 

\begin{enumerate}
\item Explain the fundamental problem of causal inference and its implications for identifying causal relationships in the social sciences
\item Explain principles of construct validity, internal validity, external validity, and statistical conclusion validity with regard to experimental design, analysis, and reporting
\item Evaluate trade-offs in the design, analysis, and reporting of experimental research and explain the implications of experimental and non-experimental designs for drawing causal inferences
\item Apply methodological and substantive knowledge from the course to the design and analysis of an original experiment
\end{enumerate}

In order to achieve and evaluate these learning objectives, students design an experiment and associated plan of analysis for that experiment. The home assignment format of the exam allows students to wrestle with and explain how broad theoretical concerns apply to their specific empirical project, thereby applying both substantive and analytical knowledge to a specific situation. By writing a long-form assignment, they also have the opportunity to explain and evaluate different design decisions, with reference to methodological literature and to published research in their specific area of interest. This course therefore is very student-focused, given that choice of topic is completely open and the only constraint is that students work within the context of experimental approach.
\end{comment}%

\subsubsection{``Quantitative Political Analysis''}

The purpose of this course is to train students in the fundamentals of quantitative analysis of political phenomena, including theory development and testing, statistical theories and methods, and the effective use of appropriate statistical software. The course therefore uses published literature addressing real-world political questions to introduce, explain, and instruct about particular methodological strategies. The intended learning outcomes for the course are that students should be able to:

\begin{enumerate*}
\item Frame politically relevant research questions 
\item Deduce observable implications from political science theories
\item Design quantitative studies that provide answers politically relevant research questions 
\item Describe statistical theories and apply those approaches in Stata
\item Apply the statistical methods to politically relevant research questions
\item Report and reflect on statistical results in written form in Danish and English
\end{enumerate*}

While the course takes a ``seminar'' format, it differs from the other described courses because it combines lectures and laboratory sessions. Each week a short lecture sets up the week's methodological approaches, after which students read relevant instructional material and applied examples of a method. Afterward, another interactive lecture reinforces concepts raised in the readings and clarifies students concerns. Finally, a laboratory-format session allows students to apply the methods they have learned each week to real-world political data. The laboratory sessions mimic the seven-day home examination, which asks students to apply knowledge from the course to describe relevant analytic approaches, apply those approaches to real data, and to analyze and reflect upon published results.


\clearpage
\subsection{Pedagogical Training}

I have been involved in the following formal pedagogical training:

\begin{itemize*}
\item Educational IT -- Go Online (Aarhus University, 2013)
\item Teaching Training Programme for Assistant Professors and Postdocs (Aarhus University, 2013)
\item New Teaching Assistant Conference (Northwestern University, 2010)
\end{itemize*}

Aside from specific training in pedagogy, I see content area expertise as a fundamental aspect of my approach to teaching. Without personal mastery of subject matter, it is it difficult or impossible to effectively communicate that subject to students. As a result, I consider my graduate education, ongoing research, and broad reading of published literature and contemporary politics to be critical elements of my continuous improvement as an educator. I feel that substantial investments in my own learning contribute far more to my pedagogy than any formal training in teaching. I therefore regularly read numerous major journals in political science, social psychology, mass communications, and statistics.

This broad scholarly grazing reflects my similarly diverse graduate education self-designed to produce exposure to substantive knowledge and methodological approaches from an array of disciplines that might benefit my political science pedagogy. To name a few, I have taken substantive courses in social psychology and communications, as well as methodological courses in sociology, statistics, communications, and psychology that offered different pedagogical approaches, literature, and learning evaluation techniques to the topic of applied statistics. These formal experiences, in additional to supplemental summer training in political psychology (at Stanford University) and in causal inference (at two joint Northwestern University/University of Southern California) workshops, provide me with exposure to a diverse set of teaching strategies and content areas.

\subsection{Future Pedagogical Activities}

While the materials included in this portfolio reflect my past teaching experiences, I also aim to enhance my future pedagogical qualifications through teaching a broader set of topics and the use of diverse forms of instruction beyond the small-scale teaching I have already performed. In terms of topic areas, I hope to continue to teach methodological courses on quantitative and qualitative research (including introductory statistical methods, regression analysis, and research design), survey design and analysis, and experimental and quasi-experimental methods. I would also like to teach substantive courses on additional areas of public opinion research (including framing and selective exposure), political communication, American politics, and specific topics in political psychology (including social cognition and motivated reasoning). Another course I am to design and teach in the near future will examine philosophical questions surrounding causation, contemporary methods for drawing causal inferences, and political controversies that center on claims about causality. Though I feel seminar style is an ideal pedagogical format, I am also interested in pursuing online, blended, and large lecture teaching formats.

In the area of methodological coursework, I am also eager to design more extensive curricula in political analysis that reflect the current trend toward identification-oriented social science. Specifically, I would like to work collaboratively to develop a coherent and comprehensive methodological sequence for bachelor, master, and doctoral students that touches on quantitative, qualitative, and philosophical issues of causation. The utility of both general tools and substantive knowledge of causation in politics and policymaking are similarly valuable for their subsequent careers and roles as democratic citizens.

I have also begun working to develop teaching materials, including innovative instructional texts and assignments, aimed at learning statistical theory and practice through the R statistical language. A working draft of these materials is publicly available at \url{http://www.thomasleeper.com/Rcourse}. More broadly, I aim in all of my teaching to produce and disseminate teaching materials for use by others, under free and open-source licenses. Toward this end, all of my current and future courses are hosted on GitHub (e.g., \url{https://github.com/leeper/opinioncourse}), which allows other teachers and students to easily copy and modify my course materials for their own purposes.

%My approach to future teaching is open-minded and pluralistic: my goal as an educator is to challenge students to think about things they have never thought about before, consider issues in new ways, and learn how to approach future challenges in an insightful and rigorous fashion. As a result, I am open teaching courses that involve nontraditional pedagogy, such as online instruction and community service-based or problem-based learning. Prior to beginning my academic career, I worked as a private sector researcher studying the effectiveness of service-learning. I would be interested in applying my research and training from that earlier career in helping students to approach political issues in a hands-on fashion.

\clearpage

\section{Student Evaluations of Teaching}

I report below quantitative and qualitative evaluations of my teaching assistant experiences from Northwestern University. Average evaluations on four questions are provided on a 1-6 scale.\\

\subsection{Quantitative Student Evaluations}
\begin{center}
\begin{tabular}{p{3.25in} p{1.25in} p{1.25in}}\toprule
\emph{Criterion}	
&\emph{Methods of Political Inference}	
&\emph{Statistical Research Methods}\\ \midrule
TA was able to answer the students' questions adequately 	&5.56	&5.71\\
TA was well prepared for each session 					&5.53	&5.57\\
TA communicated ideas in a clear manner 				&5.31	& 5.57\\
TA showed strong interest in teaching the course 			&5.53	&5.57\\
\bottomrule
\end{tabular}
\end{center}

\subsection{Selected Student Comments}
\begin{itemize*}
\item Thomas was the best TA I have ever had. No other TA has ever been so helpful, accessible, or responsive. He replied to emails quickly and with very long, detailed responses. He always extended office hours around exam/paper time. He was actually just the best TA ever. He also made discussions really interesting since he would implement activities or turn the material into an interesting conversation apart from just summarizing readings or lectures.
\item Thomas is great! He's very willing to help answer our questions and quite funny. Thomas is laid back and helped us see how to apply Methods to the real world.
\item Thomas is the PERFECT TA. Thomas likes and really, really understands PS methods. He is super prepared for section, makes jokes, brings in activities for us to do that relate to lectures, and grades with understanding. He is also just a really nice guy - you can talk to him about your papers and get help with questions, but you can also enjoy a good conversation with him in office hours. He cares about his students and teaches them a lot. It was a pleasure.
%\item Thomas was great; he expanded on lectures to make sure that we all understood what was taking place in class. Our section discussions were lively and very helpful to me.
%\item The TA was fantastic. Knew his stuff. Challenged students. Facilitated discussion. Fantastic
\item Thomas is one of the most patient, helpful, and knowledgeable TAs I've had, and he was extremely helpful during all of the class assignments--as someone who struggled with the coding and stats in this class, Thomas did a great job trying to break it down for me and explain concepts in multiple ways--analogies, explanations, alternate questions, etc. Awesome TA! He also really connects well with his students; he knew my name by the second week and emphasized personal growth and learning over the quarter, which is extremely important in classes like this where the material is very dense and easier to grasp for students with stronger math backgrounds--not always common in poli sci students. %Thomas also always answered frantic emails about the labs (which was extremely helpful) and make himself very available to meet with students even outside his office hours. A++++ in my book
%\item Thomas is great. Very knowledgeable, helped make me care about what I was learning in class, and made me feel that while it was difficult, it may be useful later. Would be happy to have him as a TA again
\end{itemize*}


\end{document}